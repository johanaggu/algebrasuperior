\documentclass[a4paper,12pt]{article}
\usepackage[utf8]{inputenc}
\usepackage[spanish]{babel}
\usepackage{graphicx}
\usepackage{geometry}
\usepackage{icomma}
\usepackage{siunitx}

% Configuración de márgenes
\geometry{top=2cm, bottom=2cm, left=2.5cm, right=2.5cm}

\begin{document}

% Portada
\begin{titlepage}
    \centering
    \includegraphics[width=0.4\textwidth]{FESAcatlanUNAMLogo.png} % Cambia el nombre de archivo por el de tu imagen
    \vspace{1cm}
    
    {\scshape\large Universidad Nacional Autónoma de México \par}
    {\large Facultad de Estudios Superiores Acatlán \par}
    \vspace{1.5cm}
    
    {\Large\bfseries Licenciatura en Matemáticas Aplicadas \par}
    \vspace{2cm}
    
    {\Huge\bfseries  \(z \cdot z^{-1} = 1 \)  \par}
    \vspace{2cm}
    
    {\Large\itshape Johan Avila Guerrero \par}
    \vfill
    
    
    \vfill
    
\end{titlepage}


% Índice

% Comienza el contenido
\section*{Probar que la multiplicacion Suma de }
\noindent 
    Para empezar definimos \(z=(a, b) \), \( z^{-1}= (\frac{a}{a^2+b^2}, \frac{-b}{a^2+b^2}  )\)  y 1 = (1,0)\\ \\
    Definicion de la operacion  \((a,b)\cdot (c,d) = (ac-bd, ad+cb)\)
    Queremos comprobar que para la pareja ordenada z = (1,3) secumpla que \(z \cdot z^{-1} = 1\), esto porque 1 es el elemento neutro en la operacion \(\cdot\)
    Procedemos a desarrollar. \\

    \(z \cdot z^{-1} = (1,3) \cdot (\frac{1}{(1)^2 + (3)^2}, \frac{-(3)}{(1)^2 + (3)^2} )  = (1,3) \cdot (\frac{1}{1 + 9}, \frac{-3}{1 + 9} ) \) \\ \\

    \(= (1,3) \cdot (\frac{1}{10}, \frac{-3}{10} ) = ( [1 (\frac{1}{10})] - [3 (-\frac{3}{10})], [1(-\frac{3}{10})] + [3(\frac{1}{10})] )\) \\\\

\(= ([\frac{1}{10}] - [-\frac{9}{10}] , [-\frac{3}{10}] + [\frac{3}{10}]  ) = (\frac{1}{10} + \frac{9}{10} , -\frac{3}{10} + \frac{3}{10}  ) \) \\\\

\(= (\frac{10}{10}  , 0  )  = (1, 0) = 1\) \\

Por lo tanto para la pareja ordenada z=(1,3) es cierto que \(z \cdot z^{-1} = 1\)

    
 

\end{document}
