\documentclass[a4paper,12pt]{article}
\usepackage[utf8]{inputenc}
\usepackage[spanish]{babel}
\usepackage{graphicx}
\usepackage{geometry}
\usepackage{icomma}
\usepackage{siunitx}
\usepackage{tikz}
\usepackage{amssymb}
\usepackage{extarrows}
\usepackage{stackengine}
\usepackage{enumitem}


% Configuración de márgenes
\geometry{top=2cm, bottom=2cm, left=2.5cm, right=2.5cm}

\begin{document}

% Portada
\begin{titlepage}
    \centering
    \includegraphics[width=0.4\textwidth]{FESAcatlanUNAMLogo.png} % Cambia el nombre de archivo por el de tu imagen
    \vspace{1cm}
    
    {\scshape\large Universidad Nacional Autónoma de México \par}
    {\large Facultad de Estudios Superiores Acatlán \par}
    \vspace{1.5cm}
    
    {\Large\bfseries Licenciatura en Matemáticas Aplicadas \par}
    \vspace{2cm}
    
    {\Huge\bfseries Demuestra que  \(|z|^{-1} = |z^{-1}| \)

    \par}
    \vspace{2cm}
    
    {\Large\itshape Johan Avila Guerrero \par}
    \vfill
    
    
    \vfill
    
\end{titlepage}


% Índice

% Comienza el contenido
\section{Demuestra que \(|z|^{-1} = |z^{-1}|\) }
Dado \(z=(a,b) \quad , \quad z^{-1} = ( \frac{a}{a^2+b^2}, \frac{-b}{a^2+b^2} )  \) y  \(|z|=\sqrt{a^2 + b^2 } \)  procedemos a desarrollar.

\[|z|^{-1} = |z^{-1}|\] 

\[|(a,b)|^{-1} = |( \frac{a}{a^2+b^2}, \frac{-b}{a^2+b^2} )|\]

\[(\sqrt{a^2 + b^2})^{-1} = \sqrt{ (\frac{a}{a^2+b^2})^2, (\frac{-b}{a^2+b^2} )^2} \] 

\[ \frac{1}{\sqrt{a^2 + b^2}}  =  \sqrt{ \frac{(a)^2}{(a^2+b^2)^2} + \frac{(-b)^2}{(a^2+b^2)^2}}  \] 

\[ \frac{1}{\sqrt{a^2 + b^2}}  =  \sqrt{ \frac{a^2}{(a^2+b^2)^2} + \frac{b^2}{(a^2+b^2)^2}}  \] 

\[ \frac{1}{\sqrt{a^2 + b^2}}  =  \sqrt{  \frac{a^2+b^2}{(a^2+b^2)^2}}  \]

\[ \frac{1}{\sqrt{a^2 + b^2}}  =  \sqrt{  \frac{a^2+b^2}{(a^2+b^2)(a^2+b^2)}}  \] 

\[ \frac{1}{\sqrt{a^2 + b^2}}  =  \sqrt{  \frac{1}{a^2+b^2}}  \] 
\[ \frac{1}{\sqrt{a^2 + b^2}}  =    \frac{\sqrt{1}}{ \sqrt{a^2+b^2}}   \] 
\[ \frac{1}{\sqrt{a^2 + b^2}}  =    \frac{1}{ \sqrt{a^2+b^2}}   \] 



Por lo tanto la afirmacion es cierta

\end{document}