\documentclass[a4paper,12pt]{article}
\usepackage[utf8]{inputenc}
\usepackage[spanish]{babel}
\usepackage{graphicx}
\usepackage{geometry}
\usepackage{icomma}
\usepackage{siunitx}
\usepackage{tikz}


% Configuración de márgenes
\geometry{top=2cm, bottom=2cm, left=2.5cm, right=2.5cm}

\begin{document}

% Portada
\begin{titlepage}
    \centering
    \includegraphics[width=0.4\textwidth]{FESAcatlanUNAMLogo.png} % Cambia el nombre de archivo por el de tu imagen
    \vspace{1cm}
    
    {\scshape\large Universidad Nacional Autónoma de México \par}
    {\large Facultad de Estudios Superiores Acatlán \par}
    \vspace{1.5cm}
    
    {\Large\bfseries Licenciatura en Matemáticas Aplicadas \par}
    \vspace{2cm}
    
    {\Huge\bfseries  \(z^{n} = (1, -1)^{n} \)  \par}
    \vspace{2cm}
    
    {\Large\itshape Johan Avila Guerrero \par}
    \vfill
    
    
    \vfill
    
\end{titlepage}


% Índice

% Comienza el contenido
\section*{Sea z = (1, −1). Calcule \(z^n\) , donde  \(n\) es un número entero positivo.}
\noindent \(z^n=(1, -1)^n  \)   \\ \\
Primero lo podemos pasar a sus coordenadas polares ya que sabemos que un numero complejo elevado a la \(n\) es igual a \((r^n, n \cdot \theta)\)

\begin{tikzpicture}[scale=1.5]
    % Ejes
    \draw[->] (-1.5,0) -- (1.5,0) node[right] {$\Re$};
    \draw[->] (0,-1.5) -- (0,1.5) node[above] {$\Im$};
    
    % Número complejo
    \draw[->,thick,red] (0,0) -- (315:1.41) node[midway, below left] {$\sqrt{2}$} node[right] {$(\sqrt{2}, 315^\circ)$};
    
    % Círculo unidad
    \draw[thin,dashed] (0,0) circle (1);
    
    % Ángulo
    \draw[->] (0.5,0) arc (0:315:0.5) node[midway, right] {$315^\circ$};
\end{tikzpicture}

Dado lo anterior sabemos que la forma polar de la pareja \( z = (1, -1)\) es \( z = (\sqrt{2}, 315^\circ)\)

Por lo tanto : \\

\(z^n = ((\sqrt{2})^n, n(315^\circ))\)

\end{document}
