\documentclass[a4paper,12pt]{article}
\usepackage[utf8]{inputenc}
\usepackage[spanish]{babel}
\usepackage{graphicx}
\usepackage{geometry}
\usepackage{icomma}
\usepackage{siunitx}
\usepackage{tikz}
\usepackage{amssymb}
\usepackage{extarrows}
\usepackage{polynom}
\usepackage{graphicx}


% Configuración de márgenes
\geometry{top=2cm, bottom=2cm, left=2.5cm, right=2.5cm}

\begin{document}

% Portada
\begin{titlepage}
    \centering
    \includegraphics[width=0.4\textwidth]{FESAcatlanUNAMLogo.png} % Cambia el nombre de archivo por el de tu imagen
    \vspace{1cm}
    
    {\scshape\large Universidad Nacional Autónoma de México \par}
    {\large Facultad de Estudios Superiores Acatlán \par}
    \vspace{1.5cm}
    
    {\Large\bfseries Licenciatura en Matemáticas Aplicadas \par}
    \vspace{2cm}
    
    {\Huge\bfseries  División de polinomios .

    \par}
    \vspace{2cm}
    
    {\Large\itshape Johan Avila Guerrero \par}
    \vfill
    
    
    \vfill
    
\end{titlepage}


% Índice

% Comienza el contenido
\section*{División de polinomios.}

Dado \(  p(x)=0x^0 + 2x^1+3x^2 \quad  \) y \(\quad d(x)= -1x^0 + 8x^1 + 1x^2 - 10x^3 \quad\) calcula \(\quad \frac{d(x)}{p(x)}\) \\





\[
\begin{array}{c|l}
  \multicolumn{2}{r}{-\frac{10}{3}x +\frac{23}{9}} \\
  \cline{2-2}
  3x^2+2x & -10x^3 + x^2 + 8x - 1 \\
  \multicolumn{2}{c}{\quad+10x^3 + \frac{20}{3}x^2 } \\ 
  
  \cline{2-2}
  
  \multicolumn{2}{c}{\qquad 0 \quad + \frac{23}{3}x^2 +8x} \\
  \multicolumn{2}{c}{\qquad \qquad - \frac{23}{3} x^2 - \frac{46}{9}x} \\
  \cline{2-2}
  \multicolumn{2}{c}{\qquad \quad 0 + \frac{26}{9}x-1}  \\
 \end{array}
\]
\\\\

Entonces tenemos que \(\frac{d(x)}{p(x)} = -\frac{10}{3}x + \frac{23}{9}+ \frac{ \frac{26}{9}x-1 }{3x^2+2x} = -\frac{10}{3}x + \frac{23}{9}+ \frac{ 26x-9 }{9(3x^2+2x)}   \) 
\end{document}
