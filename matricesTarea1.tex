\documentclass[a4paper,12pt]{article}
\usepackage[utf8]{inputenc}
\usepackage[spanish]{babel}
\usepackage{graphicx}
\usepackage{geometry}
\usepackage{icomma}
\usepackage{siunitx}
\usepackage{tikz}
\usepackage{amssymb}
\usepackage{extarrows}
\usepackage{polynom}
\usepackage{graphicx}
\usepackage{amsmath}



% Configuración de márgenes
\geometry{top=2cm, bottom=2cm, left=2.5cm, right=2.5cm}

\begin{document}

% Portada
\begin{titlepage}
    \centering
    \includegraphics[width=0.4\textwidth]{FESAcatlanUNAMLogo.png} % Cambia el nombre de archivo por el de tu imagen
    \vspace{1cm}
    
    {\scshape\large Universidad Nacional Autónoma de México \par}
    {\large Facultad de Estudios Superiores Acatlán \par}
    \vspace{1.5cm}
    
    {\Large\bfseries Licenciatura en Matemáticas Aplicadas \par}
    \vspace{2cm}
    
    {\Huge\bfseries  Matrices ejemplos de \(A^2 - B^2 \neq (A+B)(A-B)  \) y \( (A+B)^2\neq A^2+2AB+B^2  \) 

    \par}
    \vspace{2cm}
    
    {\Large\itshape Johan Avila Guerrero \par}
    \vfill
    
    
    \vfill
    
\end{titlepage}


% Índice

% Comienza el contenido
\section{\(A^2 - B^2 \neq (A+B)(A-B)  \)}
Consideremos las siguientes matrices:

\[ A = \begin{bmatrix} 1 & 2 \\ 3 & 4 \end{bmatrix} \]

y 

\[ B = \begin{bmatrix} 5 & 6 \\ 7 & 8 \end{bmatrix} \]

Calculemos \(A^2 - B^2\) y \((A+B)(A-B)\) para ver si son diferentes:

\[ A^2 = \begin{bmatrix} 1 & 2 \\ 3 & 4 \end{bmatrix} \cdot \begin{bmatrix} 1 & 2 \\ 3 & 4 \end{bmatrix} = \begin{bmatrix} 7 & 10 \\ 15 & 22 \end{bmatrix} \]

\[ B^2 = \begin{bmatrix} 5 & 6 \\ 7 & 8 \end{bmatrix} \cdot \begin{bmatrix} 5 & 6 \\ 7 & 8 \end{bmatrix} = \begin{bmatrix} 67 & 78 \\ 91 & 106 \end{bmatrix} \]

Entonces, \(A^2 - B^2\) es:

\[ A^2 - B^2 = \begin{bmatrix} 7 & 10 \\ 15 & 22 \end{bmatrix} - \begin{bmatrix} 67 & 78 \\ 91 & 106 \end{bmatrix} = \begin{bmatrix} -60 & -68 \\ -76 & -84 \end{bmatrix} \]

Ahora, calculemos \((A+B)(A-B)\):

\[ A+B = \begin{bmatrix} 1 & 2 \\ 3 & 4 \end{bmatrix} + \begin{bmatrix} 5 & 6 \\ 7 & 8 \end{bmatrix} = \begin{bmatrix} 6 & 8 \\ 10 & 12 \end{bmatrix} \]

\[ A-B = \begin{bmatrix} 1 & 2 \\ 3 & 4 \end{bmatrix} - \begin{bmatrix} 5 & 6 \\ 7 & 8 \end{bmatrix} = \begin{bmatrix} -4 & -4 \\ -4 & -4 \end{bmatrix} \]

Entonces, \((A+B)(A-B)\) es:

\[ (A+B)(A-B) = \begin{bmatrix} 6 & 8 \\ 10 & 12 \end{bmatrix} \cdot \begin{bmatrix} -4 & -4 \\ -4 & -4 \end{bmatrix} = \begin{bmatrix} -56 & -56 \\ -88 & -88 \end{bmatrix} \]

En este caso, \(A^2 - B^2\) no es igual a \((A+B)(A-B)\).


\section{\((A + B)^2 \neq A^2+2AB +B^2  \)}
Consideremos las siguientes matrices:

\[ A = \begin{bmatrix} 1 & 2 \\ 3 & 4 \end{bmatrix} \]

y 

\[ B = \begin{bmatrix} 5 & 6 \\ 7 & 8 \end{bmatrix} \]

Calculemos \((A + B)^2 \) y \(A^2+2AB+B^2\) para ver si son diferentes: \\

Calculemos \((A+B)^2\):

\[ (A+B)^2 = 
(\begin{bmatrix} 1 & 2 \\ 3 & 4 \end{bmatrix} + 
\begin{bmatrix} 5 & 6 \\ 7 & 8 \end{bmatrix} )^2 = 
\begin{bmatrix} 6 & 8 \\ 10 & 12 \end {bmatrix}^2 =
\begin{bmatrix} 116 & 144 \\ 180 & 224 \end{bmatrix} \] 

Ahora calculemos \(A^2+2AB+B^2\)

\[A^2 +2AB+B^2= 
\begin{bmatrix} 1 & 2 \\ 3 & 4 \end{bmatrix}^2 +
2\begin{bmatrix} 1 & 2 \\ 3 & 4 \end{bmatrix} \begin{bmatrix} 5 & 6 \\ 7 & 8 \end{bmatrix} +
\begin{bmatrix} 5 & 6 \\ 7 & 8 \end{bmatrix}^2 =
\]

\[
\begin{bmatrix} 7 & 10 \\ 15 & 22 \end{bmatrix} +
\begin{bmatrix} 38 & 44 \\ 86 & 100 \end{bmatrix} +
\begin{bmatrix} 67 & 78 \\ 91 & 106 \end{bmatrix} =
\begin{bmatrix} 112 & 132 \\ 192 & 228 \end{bmatrix} 
\]
\\\\
Por lo tanto \((A+B)^2 \neq A^2 +2AB+B^2\)



\end{document}
