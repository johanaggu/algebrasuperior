\documentclass[a4paper,12pt]{article}
\usepackage[utf8]{inputenc}
\usepackage[spanish]{babel}
\usepackage{graphicx}
\usepackage{geometry}
\usepackage{icomma}
\usepackage{siunitx}
\usepackage{tikz}
\usepackage{amssymb}
\usepackage{extarrows}


% Configuración de márgenes
\geometry{top=2cm, bottom=2cm, left=2.5cm, right=2.5cm}

\begin{document}

% Portada
\begin{titlepage}
    \centering
    \includegraphics[width=0.4\textwidth]{FESAcatlanUNAMLogo.png} % Cambia el nombre de archivo por el de tu imagen
    \vspace{1cm}
    
    {\scshape\large Universidad Nacional Autónoma de México \par}
    {\large Facultad de Estudios Superiores Acatlán \par}
    \vspace{1.5cm}
    
    {\Large\bfseries Licenciatura en Matemáticas Aplicadas \par}
    \vspace{2cm}
    
    {\Huge\bfseries  Conmutatividad en producto de números racionales \(\mathbb{Q}\).

    \par}
    \vspace{2cm}
    
    {\Large\itshape Johan Avila Guerrero \par}
    \vfill
    
    
    \vfill
    
\end{titlepage}


% Índice

% Comienza el contenido
\section*{Demostrar conmutatividad en la operación producto de los números racionales \(\mathbb{Q}\).}
\noindent  Definimos el producto en el conjunto de numeros racionales como \(\mathbb{Q}_{*} := [(a,b)] * [(c,d)] := [(ac , bd)] \) donde \(a,b,c,d \in \mathbb{Z}\)  \\ \\


Dado \(q_{1} = [(a,b)]\) y \(q_{2} = [(c,d)]\) , demuestra que \( q_{1}  *_{\mathbb{Q}}  q_{2} =  q_{2}*_{\mathbb{Q}} q_{1}    \) \\

\(q_{1} * q_{2}   =  [(a,b)] * [(c,d)] \xlongequal[]{ \mathbb{Q}_{*}} [(ac, bd)] \xlongequal[]{ Conmutat-\mathbb{Z}_{*}}   [(ca, db)] \xlongequal[]{ \mathbb{Q}_{*}} [(c,d)] * [(a,b)] = q_{2}*q_{1} \)



\end{document}
