\documentclass[a4paper,12pt]{article}
\usepackage[utf8]{inputenc}
\usepackage[spanish]{babel}
\usepackage{graphicx}
\usepackage{geometry}
\usepackage{icomma}
\usepackage{siunitx}
\usepackage{tikz}
\usepackage{amssymb}
\usepackage{extarrows}
\usepackage{stackengine}
\usepackage{enumitem}


% Configuración de márgenes
\geometry{top=2cm, bottom=2cm, left=2.5cm, right=2.5cm}

\begin{document}

% Portada
\begin{titlepage}
    \centering
    \includegraphics[width=0.4\textwidth]{FESAcatlanUNAMLogo.png} % Cambia el nombre de archivo por el de tu imagen
    \vspace{1cm}
    
    {\scshape\large Universidad Nacional Autónoma de México \par}
    {\large Facultad de Estudios Superiores Acatlán \par}
    \vspace{1.5cm}
    
    {\Large\bfseries Licenciatura en Matemáticas Aplicadas \par}
    \vspace{2cm}
    
    {\Huge\bfseries Examen unidad 5 

    \par}
    \vspace{2cm}
    
    {\Large\itshape Johan Avila Guerrero \par}
    \vfill
    
    
    \vfill
    
\end{titlepage}


% Índice

% Comienza el contenido
\section{Resuelve el sistema}
\[ x_1 + 2x_2 - 3x_3 = 2 \] 
\[ 2x_1 - x_2 + 2x_3 = 3 \] 
\[ 4x_1 + 3x_2 - 4x_3 = 7 \] 

Buscamos su forma escalonada usando la matriz aumentada \\ \\

    \( 
        \left[\begin{array}{ccc|c}
        1 & 2 & -3 & 2 \\
        2 & -1 & 2 & 3 \\
        4 & 3 & -4 & 7 
        \end{array}\right] 
        \xlongequal[R_3=R_3-4R_1]{R_2=R_2 - 2R_1}
        \left[\begin{array}{ccc|c}
        1 & 2 & -3 & 2 \\
        0 & -5 & 8 & -1 \\
        0 & -5 & 8 & -1 
        \end{array}\right] 
        \xlongequal[]{R_3=R_3 - R_2}
        \left[\begin{array}{ccc|c}
        1 & 2 & -3 & 2 \\
        0 & -5 & 8 & -1 \\
        0 & 0 & 0 & 0 
        \end{array}\right]
    \) \\ \\

Transformamos a un nuevo sistem de ecuaciones y despejamos los valores de \(x_1, \quad x_2, \quad x_3\)

\[x_1 + 2 x_2 - 3x_3 = 2\] 
\[-5x_2 + 8x_3 = -1 \]

Despejamos  \(x_2 \quad con \quad respecto \quad a \quad x_3\) \\ \\
\(-5x_2 +8x_3 = -1\) \\
\(-5x_2 = -1 - 8x_3  \) \\ 
\( -1 (-5x_2)= -1(-1 - 8x_3)  \) \\
\( 5x_2 = 8x_3 + 1  \) \\ \\
\(x_2 = \frac{8x_3}{5} + \frac{1}{5} \) \\\\

Despejamos \(x_1\) utilizando el valor de \(x_2\) con respecto a \(x_3\) \\\\
\(x_1+2x_2 - 3x_3 = 2\) \\ \\ 
\(x_1 + 2( \frac{8x_3 + 1}{5}) -3x_3 = 2\) \\ \\
\(x_1 + \frac{16x_3}{5} + \frac{2}{5} - 3x_3 = 2\) \\\\
\(x_1 + \frac{16x_3}{5} -3x_3 = 2 -\frac{2}{5}\) \\\\
\(x_1+\frac{x_3}{5} = \frac{8}{5}\) \\\\
\(x_1 = - \frac{x_3}{5} + \frac{8}{5}\)


\section{Determina los valores de a para que el sistema tenga...}
    \begin{itemize}[label=$\bullet$]
      \item  Una solución   
      \item Una infinidad de soluciones
      \item No tenga soluciones
    \end{itemize}

    del sistema de ecuaciones

    \[x_1+2x_2+x_3=1\]
    \[-x_1 + 4x_2 + 3x_3 = 2\]
    \[2x_1 - 2x_2+ ax_3 = 3\]

Buscamos la forma escalonada de la  sig matriz para encontrar los valores \\ \\

    \[
        \left[\begin{array}{ccc|c}
        1 & 2 & 1 & 1 \\
        -1 & 4 & 3 & 2 \\
        2 & -2 & a & 3 
        \end{array}\right] 
    \] \\  \\  


    \( 
        \left[\begin{array}{ccc|c}
        1 & 2 & 1 & 1 \\
        -1 & 4 & 3 & 2 \\
        2 & -2 & a & 3 
        \end{array}\right] 
        \xlongequal[R_2 = R2+R_1]{R_3=R_3 - 2R_1}
        \left[\begin{array}{ccc|c}
        1 & 2 & 1 & 1 \\
        0 & 6 & 4 & 3 \\
        0 & -6 & a-2 & 1 
        \end{array}\right] 
        \xlongequal[]{R_3=R_3 + R_2}
        \left[\begin{array}{ccc|c}
        1 & 2 & 1 & 1 \\
        0 & 6 & 4 & 3 \\
        0 & 0 & a+2 & 4 
        \end{array}\right] 
    \) \\ \\

    Dado lo anterior tenemos tres conclusiones

    \begin{itemize}[label=$\bullet$]
      \item  Una solución: Para tener una solucion puede tomar varios numeros pero por ejemplo tomaremos 0, lo que determinara que \(x_3 = 2\) y asi tendra una solucion exacta para \(x_1\), \(x_2\) y \(x_3\)    
      \item Una infinidad de soluciones:
      En este caso no existe ya que para hacer que sea real necesitamos que la ultima fila de la matriz en su ultima fila sea todos cero y no hay forma de que asignandole un valor a \(a\) esea verdadera la operacion
      
      \item No tenga soluciones: Para este caso tenemos que hace que la ultima fila de la matriz sea inconsistente, esto se logra con \(a = -2\) lo que nos resultara en que \(0= 4 \) y por tanto haciendo que el sistema no tenga solucion
    \end{itemize}

\section{Resuelve el sistema}
\[ w - x +y - z=1\]
\[w+x-y-z= 1\]
\[w-x-y+z=2\]
\[4w - 2x -2y = \]

Buscamos su forma escalonada reducida para encontrar una solucion al sistema

 \[
        \left[\begin{array}{cccc|c}
        1 & -1 & 1 & -1 & 1\\
        1 & 1 & -1 & -1 & 1\\
        1 & -1 & -1 & 1 & 2 \\
        4 & -2 & -2 & 0 & 1

        \end{array}\right] 
    \] \\  \\  


     \[
        \left[\begin{array}{cccc|c}
        1 & -1 & 1 & -1 & 1\\
        1 & 1 & -1 & -1 & 1\\
        1 & -1 & -1 & 1 & 2 \\
        4 & -2 & -2 & 0 & 1
        \end{array}\right] 
        \xlongequal[R_2 = R_2 -R_1   y \quad R_4= R_4 - 4R_1]{R_3=R_3 - R_1}
        \left[\begin{array}{cccc|c}
        1 & -1 & 1 & -1 & 1\\
        0 & 2 & -2 & 0 & 0\\
        0 & 0 & -2 & 2 & 1 \\
        0 & 2 & -6 & 4 & -3
        \end{array}\right] 
    \] \\  \\  

 \[
        \xlongequal[]{R_4=R_4 - R_2}
        \left[\begin{array}{cccc|c}
        1 & -1 & 1 & -1 & 1\\
        0 & 2 & -2 & 0 & 0\\
        0 & 0 & -2 & 2 & 1 \\
        0 & 0 & -4 & 4 & -3
        \end{array}\right] 
        \xlongequal[]{R_4=R_4 - 2R_3}
        \left[\begin{array}{cccc|c}
        1 & -1 & 1 & -1 & 1\\
        0 & 2 & -2 & 0 & 0\\
        0 & 0 & -2 & 2 & 1 \\
        0 & 0 & 0 & 0 & -5
        \end{array}\right] 
    \] \\  \\  


Dado que no resulta en una insonsistencia al pasarlo a un nuevo sistema de ecuaciones podemos concluir que el sistema original no tiene solucion




\end{document}