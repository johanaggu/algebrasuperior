\documentclass[a4paper,12pt]{article}
\usepackage[utf8]{inputenc}
\usepackage[spanish]{babel}
\usepackage{graphicx}
\usepackage{geometry}
\usepackage{icomma}
\usepackage{siunitx}
\usepackage{tikz}
\usepackage{amssymb}
\usepackage{extarrows}


% Configuración de márgenes
\geometry{top=2cm, bottom=2cm, left=2.5cm, right=2.5cm}

\begin{document}

% Portada
\begin{titlepage}
    \centering
    \includegraphics[width=0.4\textwidth]{FESAcatlanUNAMLogo.png} % Cambia el nombre de archivo por el de tu imagen
    \vspace{1cm}
    
    {\scshape\large Universidad Nacional Autónoma de México \par}
    {\large Facultad de Estudios Superiores Acatlán \par}
    \vspace{1.5cm}
    
    {\Large\bfseries Licenciatura en Matemáticas Aplicadas \par}
    \vspace{2cm}
    
    {\Huge\bfseries Representación polar de un número complejo \(\mathbb{C}\).

    \par}
    \vspace{2cm}
    
    {\Large\itshape Johan Avila Guerrero \par}
    \vfill
    
    
    \vfill
    
\end{titlepage}


% Índice

% Comienza el contenido
\section*{Obtén la representación polar del número complejo 4-2i y obtén la representación algebraica del número (3, 120°).}
\section{Obtén la representación polar del número complejo 4-2i.}

Primero representemos el número complejo en el plano de argand y dibujemos un triángulo rectángulo que ayude a calcular los valores del ángulo y el radio polar del número complejo.


\begin{tikzpicture}
    % Ejes
    \draw[->] (-1,0) -- (5,0) node[right] {$\mathbb{R}$};
    \draw[->] (0,-3) -- (0,1) node[above] {$\mathbb{I}$};
    
    % Vector (4, 36)
    \draw[->, thick] (0,0) -- (333.43:4.4721) node[midway, above left] {};
    
    % Punto (4,3)
    \fill (4,-2) circle[radius=2pt] node[above right] {$4 -2i$};
    
    % triangulo
    % Definición de vértices
    \coordinate[label=left:] (A) at (0,0);
    \coordinate[label=right:] (B) at (4,0);
    \coordinate[label=above:] (C) at (4,-2);
    
    % Dibujar el triángulo con lados punteados
    \draw[dashed] (A) -- (B) -- (C) -- cycle;
    
    % Marcar el ángulo recto
    \draw (3.5,0) -- (3.5,-0.5)-- (4,-0.5) ;
    
    % Etiquetar el ángulo recto
    \node at (3,-0.6) {$90^\circ$};

   
\end{tikzpicture}

Dado lo anterior podemos calcular el radio polar que está dado por \(r = \sqrt{4^2 + 2^2}\) y el angulo lo podemos calcular usando la razon trigonometrica \(tan\theta = \frac{co}{h}\), que despejando quedaria como \( \theta = tan^-1(\frac{co}{h}) \) 

\end{document}