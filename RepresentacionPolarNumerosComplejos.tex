\documentclass[a4paper,12pt]{article}
\usepackage[utf8]{inputenc}
\usepackage[spanish]{babel}
\usepackage{graphicx}
\usepackage{geometry}
\usepackage{icomma}
\usepackage{siunitx}
\usepackage{tikz}
\usepackage{amssymb}
\usepackage{extarrows}
\usepackage{pgfplots}
\pgfplotsset{compat=1.17}
\usepackage{tikz}
\usetikzlibrary{angles,quotes}


% Configuración de márgenes
\geometry{top=2cm, bottom=2cm, left=2.5cm, right=2.5cm}

\begin{document}

% Portada
\begin{titlepage}
    \centering
    \includegraphics[width=0.4\textwidth]{FESAcatlanUNAMLogo.png} % Cambia el nombre de archivo por el de tu imagen
    \vspace{1cm}
    
    {\scshape\large Universidad Nacional Autónoma de México \par}
    {\large Facultad de Estudios Superiores Acatlán \par}
    \vspace{1.5cm}
    
    {\Large\bfseries Licenciatura en Matemáticas Aplicadas \par}
    \vspace{2cm}
    
    {\Huge\bfseries Representación polar de un número complejo \(\mathbb{C}\).

    \par}
    \vspace{2cm}
    
    {\Large\itshape Johan Avila Guerrero \par}
    \vfill
    
    
    \vfill
    
\end{titlepage}


% Índice

% Comienza el contenido
\section*{Obtén la representación polar del número complejo 4-2i y obtén la representación algebraica del número (3, 120°).}
\section{Obtén la representación polar del número complejo 4-2i.}

Primero representemos el número complejo en el plano de argand y dibujemos un triángulo rectángulo que ayude a calcular los valores del ángulo y el radio polar del número complejo.


\begin{tikzpicture}
    % Ejes
    \draw[->] (-1,0) -- (5,0) node[right] {$\mathbb{R}$};
    \draw[->] (0,-3) -- (0,1) node[above] {$\mathbb{I}$};
    
    % Vector (4, 36)
    \draw[->, thick] (0,0) -- (333.43:4.4721) node[midway, above left] {};
    
    % Punto (4,3)
    \fill (4,-2) circle[radius=2pt] node[above right] {$4 -2i$};
    
    % triangulo
    % Definición de vértices
    \coordinate[label=left:] (A) at (0,0);
    \coordinate[label=right:] (B) at (4,0);
    \coordinate[label=above:] (C) at (4,-2);
    
    % Dibujar el triángulo con lados punteados
    \draw[dashed] (A) -- (B) -- (C) -- cycle;
    
    % Marcar el ángulo recto
    \draw (3.5,0) -- (3.5,-0.5)-- (4,-0.5) ;
    
    % Etiquetar el ángulo recto
    \node at (3,-0.6) {$90^\circ$};

   
\end{tikzpicture}

Considerando el punto \(M(x,y)\)  donde  \(x= 4\)  y \(y=-2\) podemos calcular el radio polar que está dado por \(r = \sqrt{x^2 + y^2}\) y el angulo lo podemos calcular usando \( \theta = arctan(\frac{y}{x}) + 2\pi \) para todo \(x > 0\) y \( y < 0\).
Entonces podemos calcular el valor del radio y ángulo del número de la siguiente forma.

Cálculo del radio: 
\[r = \sqrt{4^2 + -2^2} \quad,\quad r = 2\sqrt{5}\]

Cálculo del ángulo:
\[\theta = arctan(\frac{y}{x}) + 2\pi \quad , \quad \theta = arctan(\frac{-2}{4}) + 2\pi \]
\[\theta = 333_\cdot44^\circ \]

Por lo anterior la representacion polar del numero complejo \(4-2i\) es \((2\sqrt{5}, 333_\cdot44^\circ 
 ) \)

\begin{tikzpicture}
  % Eje x
  \draw[->] (0,0) -- (5,0) node[right] {$\Re$};
  % Eje y
  \draw[->] (0,0) -- (0,5) node[above] {$\Im$};
  % Línea radial
  \draw[red, thick] (0,0) -- (333.4:4);
  % Ángulo
  \draw[blue, thick, ->] (0.7,0) arc (0:333.4:0.7);
  \draw (1,0.3) node[blue] {$333.4^\circ$};
  % Punto
  \node[label={333.4:$(2\sqrt{5}, 333_\cdot44^\circ)$},circle,fill,inner sep=2pt,red] at (333.4:4) {};
\end{tikzpicture}


\section{Obtén la representación algebraica del número (3, 120°).}.
Primero representemos visualmente el numero, que es el siguiente.

\begin{tikzpicture}
  % Eje x
  \draw[->] (0,0) -- (4,0) node[right] {$x$};
  % Eje y
  \draw[->] (0,0) -- (0,4) node[above] {$y$};
  % Línea radial
  \draw[red, thick] (0,0) -- (120:3);
  % Ángulo
  \draw[blue, thick, ->] (0.8,0) arc (0:120:0.8);
  \draw (1,0.3) node[blue] {$3, 120^\circ$};
\end{tikzpicture}

La representacion algebraica de un numero complejo se puede obtener como \(x = hCos\theta \) y \(y =hSin\theta \) donde \(h=3 \) y\( M(x,y)\) , siguiendo lo anterior podemos calcularlo como:

\[x = hCos\theta \quad , \quad x=3Cos120^\circ\]
\[x=-\frac{3}{2}\]

\[y = hSin\theta \quad , \quad y=3Sin120^\circ\]
\[y=\frac{3\sqrt{3}}{2}\]

Por lo tanto la representacion algebraica de \((3, 120^\circ)\) es \(-\frac{3}{2} + \frac{3\sqrt{3}}{2}i\)

\begin{tikzpicture}
    % Ejes
    \draw[->] (-5,0) -- (1,0) node[right] {$\mathbb{R}$};
    \draw[->] (0,-1) -- (0,5) node[above] {$\mathbb{I}$};
    
    % Vector (4, 36)
    \draw[->, thick] (0,0) -- (120:3) node[midway, above left] {};
    
    % Punto (4,3)
    \fill (-1.5,2.5980) circle[radius=2pt] node[above right] {$-\frac{3}{2} + \frac{3\sqrt{3}}{2}i$};
    
    % triangulo
    % Definición de vértices
    
    % \coordinate[label=left:] (A) at (0,0);
    % \coordinate[label=right:] (B) at (4,0);
    % \coordinate[label=above:] (C) at (4,-2);
    
    % Dibujar el triángulo con lados punteados
    % \draw[dashed] (A) -- (B) -- (C) -- cycle;
    
    % Marcar el ángulo recto
    
    % Etiquetar el ángulo recto

   
\end{tikzpicture}

\end{document}